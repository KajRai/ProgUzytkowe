\documentclass[a4paper]{article}
\usepackage[left=3.5cm, right=2.5cm, top=2.5cm, bottom=2.5cm]{geometry}
\usepackage[MeX]{polski}
\usepackage[utf8]{inputenc}
\usepackage{graphicx}
\usepackage{enumerate}
\usepackage{amsmath} %pakiet matematyczny
\usepackage{amssymb} %pakiet dodatkowych symboli
\usepackage{multirow}
\usepackage{caption}
\captionsetup[table]{skip=10pt}
\title{TABELE}
\author{Kajetan Rainko}
\date {listopad 2022}
\begin{document}
\maketitle
\newpage
\begin{table}[h]
\centering\caption{Przykładowy system decyzyjny ($U,A,d$) modelujący problem diagnozy medycznej, której efektem jest decyzja o wykonaniu lub niewykonaniu operacji wycięcia wyrostka robaczkowego, $U=\{u_{1}, u_{2},.....,u_{10}\}$, $A=\{\alpha_{1},\alpha_{2}\}$, $d \in D=\{\textit{TAK,NIE}\}$}
\begin{tabular}{cccc}
\hline
\hline
Pacjent & Ból brzucha & Temperatura ciała & Operacja\\
\hline
u1 & mocny & wysoka & Tak\\
u2 & Średni & Wysoka & Tak \\
u3 & mocny & Średnia & Tak \\
u4 & mocny & Niska & Tak \\
u5 & Średni & Średnia & Tak \\
u6 & Średni & Średnia & Nie \\
u7 & Mały & wysoka & Nie \\
u8 & Mały & Niska & Nie \\
u9 & mocny & Niska & Nie \\
u10 & Mały & Średnia& Nie \\
\hline
\hline
\end{tabular}
\end{table}
\begin{table}[h]
\centering\caption{Podstawowe bramki logiczne}
\begin{tabular}{c|c|c}
\hline
\hline
Funkcja & Operator & Opis\\
\hline
NOT,INVERTER & $C=A’ $& C jest jeden jeżeli A jest 0 \\
AND & $C=A*B $& C jest jeden jeżeli A i B są jeden  \\
OR & $C=A+B $& C jest jeden jeżeli A lub B są jeden \\
XOR & $C=A \oplus B $& C jest jeden jeżeli albo A albo B jest jeden. \\
NAND &$ C=A\uparrow B $& C jest jeden jeżeli A lub B są zero  \\
NOR &$ C=A\downarrow B$ & C jest jeden jeżeli A i B są zero \\
BUF &$ C=A\equiv B $& C jest jeden jeżeli A i B są takie same \\

\hline
\hline
\end{tabular}
\end{table}
\end {document}