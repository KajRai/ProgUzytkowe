\documentclass[12pt; a4paper; titlepage]{article}
\usepackage[left=3.5cm, right=2.5cm, top=2.5cm, bottom=2.5cm]{geometry}
\usepackage[MeX]{polski}
\usepackage[utf8]{inputenc}
\usepackage{graphicx}
\usepackage{enumerate}
\usepackage{amsmath} %pakiet matematyczny
\usepackage{amssymb} %pakiet dodatkowych symboli
\title{pierwej haraszo dok w lateksie}
\author{Kajetan Rainko}
\date {pazdziernik 2022}
\begin{document}
\maketitle
\newpage
\section{Pierwsza selekcja}
"Lorem ipsum dolor sit amet, consectetur adipiscing elit, sed do eiusmod tempor incididunt ut labore et dolore magna aliqua. Ut enim ad minim veniam, quis nostrud exercitation ullamco laboris nisi ut aliquip ex ea commodo consequat. Duis aute irure dolor in reprehenderit in voluptate velit esse cillum dolore eu fugiat nulla pariatur. Excepteur sint occaecat cupidatat non proident, sunt in culpa qui officia deserunt mollit anim id est laborum."
\subsection{Podsekcja}
"Lorem ipsum dolor sit amet, consectetur adipiscing elit, sed do eiusmod tempor incididunt ut labore et dolore magna aliqua. Ut enim ad minim veniam, quis nostrud exercitation ullamco laboris nisi ut aliquip ex ea commodo consequat. Duis aute irure dolor in reprehenderit in voluptate velit esse cillum dolore eu fugiat nulla pariatur. Excepteur sint occaecat cupidatat non proident, sunt in culpa qui officia deserunt mollit anim id est laborum."
\subsubsection{Podpodpodposekcja}
"Lorem ipsum dolor sit amet, consectetur adipiscing elit, sed do eiusmod tempor incididunt ut labore et dolore magna aliqua. Ut enim ad minim veniam, quis nostrud exercitation ullamco laboris nisi ut aliquip ex ea commodo consequat. Duis aute irure dolor in reprehenderit in voluptate velit esse cillum dolore eu fugiat nulla pariatur. Excepteur sint occaecat cupidatat non proident, sunt in culpa qui officia deserunt mollit anim id est laborum."
\newpage
\begin{enumerate}
\item punkt pierwejszy
\item punkt drugejszy
\item punkt trejszy
\end{enumerate}
blablalbablalbl
\newline
\begin{enumerate}[a)]
\item pierwszaliterka
\item drugaliterka
\end{enumerate}
\newpage
\section {Rys historyczny}
\hspace{15pt} Klauzula rebus sic stantibus jest chyba jedną z najbardziej spornych i kontrowersyjnych zasad prawa międzynarodowego. Jej geneza jest bardzo dyskusyjna. Pierwsze
nawiązania do tej zasady można spotkać już w  pismach Cycerona, lecz stanowcza
większość badaczy uważa, że reguła ta nie była wówczas znanaJej pierwowzór upatruje się w pracach średniowiecznych glosatorów lub w dziełach filozofów. Za twórcę
terminu rebus sic stantibus uważany jest francuski prawnik André Tiraqueau, żyjący
w XVI w.
\textbf {Istnieją jednak głosy, że twórcami tego terminu są glosatorzy działający na
terenie dzisiejszych Włoch w XII w.} Początkowo całkiem inaczej pojmowano tę klauzulę.
Definiowano ją w ten sposób, że wszelki stosunek prawny może zgasnąć lub ulec zmianie, o ile wymaga tego zaszła zmiana stosunków. Zasada ta z upływem lat stała się
powszechną koncepcją prawa prywatnego, a w pracach wybitnych postaci epoki można
znaleźć teorie, które posłużyły do rozwoju zasadniczej zmiany okoliczności na arenie
międzynarodowej. Niccolò Machiavelli wspomina o tym, że władca powinien kierować
się zmianami losu, a o zmianie okoliczności pisali również 
\textit{–Baruch Spinoza, Thomas Hobbes oraz Emmer de Vattel}. Rozwój tej normy i okazane jej zainteresowanie przez
dyplomatów wynikały głównie z potrzeb życia międzynarodowego oraz z konieczności
zahamowania swobody odrzucania istniejących zobowiązań międzynarodowych.
Na przestrzeni dziejów z klauzuli tej korzystała m.in. Francja, Szwajcaria, Rosja oraz
Polska w dwudziestoleciu międzywojennym. Wiek oświecenia w Europie rozpoczął falę
krytyki wobec zastosowania tej zasady, co poskutkowało dominowaniem zasady pacta
sunt servanda w prawie prywatnym w XIX w.
\section{Zasadnicza zmiana okoliczności w prawie traktatów}
\hspace{15pt} Polskie prawo było jednym z  pierwszych, które uregulowało klauzulę rebus sic
stantibus na gruncie prawa prywatnego. Uregulowania te były wzorem dla innych
państw europejskich, a ich odmiany zawiera prawo angielskie, włoskie i niemieckie11.
W wyniku tego wielu znawców prawa międzynarodowego reprezentuje stanowisko
traktowania normy \textit{rebus sic stantibus} jako ogólnej zasady prawa, w wyniku czego jest
\end{document}